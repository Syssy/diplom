\documentclass{beamer}

\usepackage[latin1]{inputenc}
\usepackage[ngerman]{babel}
%\usepackage{amsmath,amssymb,amsthm}
%\usepackage{wasysym}

%\usepackage{ae}
\usepackage[T1]{fontenc}
%\usepackage{german}

\usetheme{Singapore}
\useinnertheme{circles}
\usecolortheme [cmyk={0.57,0,1,0}]{structure} 
\usecolortheme{rose}
\usefonttheme{professionalfonts}
\setbeamertemplate{frametitle}[default][left]
\setbeamertemplate{footline}[frame number]

\usepackage{dsfont}
\usepackage{multicol}

\renewcommand{\Tiny}{\fontsize{5pt}{6pt}}

\usepackage{tikz}
\usetikzlibrary{arrows,%
                petri,%
                topaths}%
\tikzstyle{knoten}=[draw,-,thick,fill=none,inner sep=0pt, minimum width=15pt, circle]
\tikzstyle{kante}=[draw,-,thick,black]
\usetikzlibrary{arrows,decorations.pathmorphing,backgrounds,positioning,fit,petri}

\definecolor{tugreen}{cmyk}{0.57,0,1,0}

\title{Simulation einer Multikapillars�ule}
\subtitle{Einf�hrungsvortrag Diplomarbeit}
\author[E.B�hmer]{Elisabeth B�hmer}
\date{\today}
\institute [TuDO] {Technische Universit�t Dortmund}

\begin{document}

\frame[plain]{
	\titlepage
}


%\uncover<4->{
%[<+->]

\frame {
	\frametitle{Gliederung}
	\tableofcontents
	%[pausesections]
	}
%\AtBeginSection[]{
%\frame {
%\frametitle{�berblick}
%\tableofcontents[current, currentsection]
%}
%}

\section{Einleitung}

%\begin{frame}
%\frametitle {}
%
%\end {frame}

\begin{frame}
\frametitle {Titel}
\end {frame}

\section{Gaschromatographie}


\begin{frame}
\frametitle {noch ein titel}

\end {frame}

\section{Modell}

\begin{frame}
\frametitle {noch ein titel}

\end {frame}


%\begin{frame}
%\frametitle {}
%
%\end {frame}
%\section{}




\begin{frame}
\frametitle {Zusammenfassung}

\end {frame}

\end{document}
