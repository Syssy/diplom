\documentclass{beamer}

\hypersetup{
  pdfstartpage=1,
  pdfpagelayout=SinglePage,
  pdfview = fit,
  pdfstartview = Fit
}

\usepackage[latin1]{inputenc}
\usepackage[ngerman]{babel}
\usepackage{amsmath,amssymb,amsthm,latexsym}
\usepackage{wasysym}

\usepackage[T1]{fontenc}

\usetheme{Singapore}
\useinnertheme{circles}
\usecolortheme [cmyk={0.57,0,1,0}]{structure} 
\usecolortheme{rose}
\usefonttheme{professionalfonts}
\setbeamertemplate{frametitle}[default][left]
\setbeamertemplate{footline}[frame number]
\setbeamertemplate{itemize item}[ball]
\setbeamertemplate{itemize subitem}[triangle]
\setbeamertemplate{itemize subsubitem}[square]

%f�r overfull boxes:
%\usepackage{showframe}

%\usepackage{dsfont}
\usepackage{multicol}

\renewcommand{\Tiny}{\fontsize{5pt}{6pt}}

\usepackage{tikz}
\tikzstyle{knoten}=[draw,-,thick,fill=none,inner sep=0pt, minimum width=15pt, circle]
\tikzstyle{kante}=[draw,-,thick,black]
\usetikzlibrary{arrows,decorations.pathmorphing,backgrounds,positioning,fit,petri}
\usetikzlibrary{shapes.geometric}
\usetikzlibrary{%
	calc,%
	decorations.pathmorphing,%
	fadings,%
	shadings}
	
	
\definecolor{tugreen}{cmyk}{0.57,0,1,0}

\title{Simulation einer Multikapillars�ule}
\subtitle{Einf�hrungsvortrag Diplomarbeit}
\author[E.B�hmer]{Elisabeth B�hmer}
\date{\today}
\institute [TuDO] {Technische Universit�t Dortmund}



\begin{document}

\frame[plain]{
	\titlepage
}

% F�r Overlays: 
%\uncover<4->{
%[<+->]

\frame {
	\frametitle{Gliederung}
	\tableofcontents
	%[pausesections]
	}
%\AtBeginSection[]{
%\frame {
%\frametitle{�berblick}
%\tableofcontents[current, currentsection]
%}
%}

\section{Einleitung}

%\begin{frame}
%\frametitle {}
%
%\end {frame}

\begin{frame}
\frametitle {PAA f�r das 2-Parameter Modell}

\begin{tikzpicture}[->,>=stealth',shorten >=1pt,auto,node distance=3cm, thick,
  state node/.style={circle,fill=white,draw,font=\sffamily\bfseries, minimum height=30pt, text width = 1.6cm, align = center},
  operation node/.style= {regular polygon, regular polygon sides=3,  fill = white, draw, inner sep=  0pt},
  emission node/.style={rectangle, fill = lightgray!50, draw, text width = 1.5cm}]

       % \draw[fill=green] (current page.north west) rectangle (current page.south east);
%  
  \node[state node] (s) at (2, 1)  {station�r};
  \node[state node] (m) [right = of s] {mobil};
  \node[operation node, align = center] (os)  at (1.5,0.2) {$+$};
  \node[operation node, align = center] (om)  at (7.5,0.2) {$+$};

  \path[every node/.style={font=\sffamily\large}]

    (s)   edge [bend right] node [below] {$1-p_s$} (m)
            edge [loop left] node {$p_s$} (s)
        
    (m)  edge [bend right] node[above]{$1-p_m$} (s)
             edge [loop right] node {$p_m$} (m)
        ;

  \fill[color=lightgray!40] (1.1, 3.5) --(2,1.5) -- (2.9,3.5) -- cycle ;
  \fill[color=lightgray!40] (6.05, 3.5) -- (6.95,1.5) -- (7.85,3.5) -- cycle ;

  \node[emission node, align = center] (em) [above of = m] {$e(0) = 0$\\$e(1)=1$};
  \node[emission node, align = center] (es) [above of = s] {$e(0) = 1$\\$e(1)=0$};
  
\end{tikzpicture}


\end {frame}

\section{Gaschromatographie}


\begin{frame}
\frametitle{Prinzip der Gaschromatographie} 

\begin{itemize}
 \item Verfahren zur Auftrennung von Stoffgemischen
 \item Verteilung der Analyten zwischen mobiler und station�rer Phase
 \item Varianten:
  \begin{itemize}
   \item Fl�ssigchromatographie
   \item Gaschromatographie
    \begin{itemize}
     \item Gepackte S�ulen
     \item Kapillars�ulen
    \end{itemize}

  \end{itemize}

\end{itemize}

 
\end{frame}


\begin{frame}
\frametitle {Prinzip der Gaschromatographie}

\begin{tikzpicture}[->,>=stealth',shorten >=1pt,auto, thick,
  point/.style={circle},
  analyte/.style={circle,fill=black,draw,minimum height=10pt},
  mobile/.style={circle,fill=white,draw,minimum height=10pt },
  stationary/.style={rectangle,fill = lightgray!50,minimum height=1.8cm}]

  %Station�re Phase
   \fill[color = lightgray!50] (-0.5,-1.5) -- (8,-1.5) -- (8,0.5) -- (-0.5,0.5) -- cycle;
   \node[stationary, text width = 15em,text centered] (stat) at (4, -0.5) {station�re Phase};
   \node[mobile](a0) at (0, 3){};
   \node[text width =5cm] at (2.8, 3) {mobile Phase};
   \node[analyte] at (3, 3){};
   \node[text width =5cm] at (5.8, 3) {Analyt};
   
\onslide<1>{
  % Start beider Teilchen
   \node[analyte](a1) at (-1, 1){};
   \draw[->](a1)--(0,1);
   \node[mobile](m1) at (-1, 1.5){};
   \draw[->](m1)--(0, 1.5);
}\onslide<2>{
   \node[analyte](a2a) at (0, 1){};
   \draw[->](a2a)--(1, 1);
   \node[mobile](m2a) at (0, 1.5){};
   \draw[->](m2a)--(1, 1.5) ;
}\onslide<3>{
   %Analyt bleibt h�ngen (L�sung)
   \node[analyte](a2) at (0.5, 0.5){};
   \draw[->](a2)--(0.5, 0) ;
   \node[mobile](m2) at (1, 1.5){};
   \draw[->](m2)--(2, 1.5) ;
}\onslide<4>{
   %L�sung, Teil2
   \node[analyte](a3) at (0.5, 0){};
   \node[mobile](m3) at (2, 1.5){};  
   \draw[->](m3)--(3, 1.5);
   \node[text width = 2cm] at (1, -2) {L�sung}; 
}\onslide<5>{
   % Weiter
   \node[analyte](a4) at (1.5, 1){};
   \draw[->](a4)--(2.5,1);
   \node[mobile](m4) at (3, 1.5){};
   \draw[->](m4)--(4, 1.5); 
}\onslide<6>{   
% Weiter
   \node[analyte](a4) at (2.5, 1){};
   \draw[->](a4)--(3.5,1);
   \node[mobile](m4) at (4, 1.5){};
   \draw[->](m4)--(5, 1.5);
}\onslide<7>{
   % Analyt bleibt h�ngen (Adsorption)
   \node[analyte](a4) at (3, 0.5){};
   \node[mobile](m4) at (5, 1.5){};
   \draw[->](m4)--(6, 1.5);
   \node[text width = 2cm] at (3, -2) {Adsorption}; 
}\onslide<8>{
   % Weiter
   \node[analyte](a4) at (4, 1){};
   \draw[->](a4)--(5,1);
   \node[mobile](m4) at (6, 1.5){};
   \draw[->](m4)--(7, 1.5);
}
\end{tikzpicture}

\end {frame}

\section{Modell}

\begin{frame}
\frametitle {noch ein titel}

\end {frame}


%\begin{frame}
%\frametitle {}
%
%\end {frame}
%\section{}




\begin{frame}
\frametitle {Zusammenfassung}

\end {frame}

\end{document}
