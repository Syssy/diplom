\documentclass{beamer}

\usepackage[latin1]{inputenc}
\usepackage[ngerman]{babel}
%\usepackage{amsmath,amssymb,amsthm}
%\usepackage{wasysym}

%\usepackage{ae}
\usepackage[T1]{fontenc}
%\usepackage{german}

\usetheme{Singapore}
\useinnertheme{circles}
\usecolortheme [cmyk={0.57,0,1,0}]{structure} 
\usecolortheme{rose}
\usefonttheme{professionalfonts}
\setbeamertemplate{frametitle}[default][left]
\setbeamertemplate{footline}[frame number]

\usepackage{dsfont}
\usepackage{multicol}

\renewcommand{\Tiny}{\fontsize{5pt}{6pt}}

\usepackage{tikz}
\usetikzlibrary{arrows,%
                petri,%
                topaths}%
\tikzstyle{knoten}=[draw,-,thick,fill=none,inner sep=0pt, minimum width=15pt, circle]
\tikzstyle{kante}=[draw,-,thick,black]
\usetikzlibrary{arrows,decorations.pathmorphing,backgrounds,positioning,fit,petri}
\usetikzlibrary{arrows,decorations.pathmorphing,backgrounds,positioning,fit}
\usetikzlibrary{shapes.geometric}
\usetikzlibrary{%
	calc,%
	decorations.pathmorphing,%
	fadings,%
	shadings}

\definecolor{tugreen}{cmyk}{0.57,0,1,0}

\title{Simulation einer Multikapillars�ule}
\subtitle{Einf�hrungsvortrag Diplomarbeit}
\author[E.B�hmer]{Elisabeth B�hmer}
\date{\today}
\institute [TuDO] {Technische Universit�t Dortmund}

\begin{document}

\frame[plain]{
	\titlepage
}

% F�r Overlays: 
%\uncover<4->{
%[<+->]

\frame {
	\frametitle{Gliederung}
	\tableofcontents
	%[pausesections]
	}
%\AtBeginSection[]{
%\frame {
%\frametitle{�berblick}
%\tableofcontents[current, currentsection]
%}
%}

\section{Einleitung}

%\begin{frame}
%\frametitle {}
%
%\end {frame}

\begin{frame}
\frametitle {Titel}

\begin{tikzpicture}[->,>=stealth',shorten >=1pt,auto,node distance=3cm, thick,
  state node/.style={circle,fill=white,draw,font=\sffamily\bfseries, minimum height=30pt},
  operation node/.style= {regular polygon, regular polygon sides=3,  fill = white, draw, inner sep=  0pt},
  emission node/.style={rectangle, fill = lightgray!50, draw}]

  \fill[color=lightgray!40] (8,2.7) -- (7, 4.8) -- (9,4.8) -- cycle ;    
  \fill[color=lightgray!40] (2,2.7) -- (1, 4.8) -- (3,4.8) -- cycle ;    

  \node[state node, text width = 1.5cm, align = center] (s) at (2, 1)  {station�r};
  \node[state node, text width = 1.5cm, align = center] (m) [right = of s] {mobil};
  \node[operation node, align = center] (os)  at (0.5,0) {$+$};
  \node[operation node, align = center] (om)  at (9.5,0) {$+$};
  \node[emission node, align = center] (em) [above of = m] {$e(0) = 0$\\$e(1)=1$};
  \node[emission node, align = center] (es) [above of = s] {$e(0) = 1$\\$e(1)=0$};

  \path[every node/.style={font=\sffamily\large}]

    (s)   edge [bend right] node [below] {$1-p_s$} (m)
            edge [loop left] node {$p_s$} (s)
        
    (m)  edge [bend right] node[above]{$1-p_m$} (s)
             edge [loop right] node {$p_m$} (m)
        ;
\end{tikzpicture}


\end {frame}

\section{Gaschromatographie}


\begin{frame}
\frametitle {noch ein titel}

\end {frame}

\section{Modell}

\begin{frame}
\frametitle {noch ein titel}

\end {frame}


%\begin{frame}
%\frametitle {}
%
%\end {frame}
%\section{}




\begin{frame}
\frametitle {Zusammenfassung}

\end {frame}

\end{document}
