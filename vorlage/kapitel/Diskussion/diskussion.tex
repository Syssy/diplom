% diskussion.tex
\chapter{Diskussion}
\label{chapter:dis}
\todo{Kapitel Diskussion schreiben}

\section{Ergebnisse}
Es wurde ein Zusammenhang zwischen den Simulationsparametern und Peakdaten gefunden.


\section{Ausblick}

% Bessere bzw. genauere Aufarbeitung der Referenzdaten: Methoden, die ztatsächlichen Peaks gut zu isolieren (\todo{kann man hier auf bestehende Arbeiten zur peakfindung verweisen?}) evtl. feinere Daten durch schätzen weiterer zwischenmessungen, um glattere Daten zu erhalten? Verfahren umd zu erkennen, ob Peaks gut sind für den Vergleich
Das ganze mit existierenden Daten vergleichen, um zu gucken, ob schmalere Peaks notwendig sind und alle realen Peakdatenkombinationen erreicht werden können, dann eventuell noch mal das Modell anpassen:

Aufgreifen der nicht umgesetzten Modellerweiterungen: Berücksichtigung des Gleichgewichts der Phasen und beschränkte Anzahl freier Plätze bei der Adsorption.

Modellideen aus \ref{chapter:mod}

Eine weitere Möglichkeit, das Modell zu verändern, besteht darin, das Gleichgewicht, welches sich zwischen den beiden Phasen aufbaut, zu berücksichtigen. Damit müssten die Wahrscheinlichkeiten, den Zustand zu wechseln, nicht mehr fest vorgegeben sein, sondern sich dynamisch während der Simulation aus der aktuellen Verteilung der Teilchen an einem bestimmten Ort auf die Zustände berechnen.

Außerdem existieren Idee, noch eine Sättigung der freien Plätze zur Adsorption einzubauen

Das ganze auch für spätere Zeitpunkte überprüfen (echte Messlänge, dauert nur halt lange mit der Sim, evtl dafür die eventsim nutzbar?)


Falls Formel für entsprecung nicht gefunden: Das Sekantenverfahren erwähnen?

