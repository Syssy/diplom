% einleitung.tex
\label{chapter:ein}

\chapter{Einleitung}

%\section{Motivation und Hintergrund}

Die Chromatographie ist ein seit langem eingesetztes Verfahren in der Chemie zur Auftrennung von Stoffgemischen. 
Die Auftrennung erfolgt dabei zwischen zwei sogenannten Phasen, der stationären und der mobilen Phase, welche sich in unterschiedlichen Aggregatzuständen befinden und untereinander nicht mischen. Die Teilchen der Stoffgemische interagieren während des chromatographischen Prozesses mit den beiden Phasen, indem sie sehr häufig zwischen den Phasen wechseln. Durch unterschiedliche Eigenschaften interagieren die Moleküle verschiedener Stoffe unterschiedlich mit den Phasen und trennen sich dadurch von den Molekülen anderer Stoffe. 

Eine mögliche Anwendung der Chromatographie ist die Präparation, bei der aus dem Stoffgemisch eine bestimmte Substanz isoliert werden soll. Zum Beispiel kann so aus einem verunreinigten Stoff, für die weitere Nutzung gereinigt werden, wenn der Stoff in Reinform zur weiteren Verwendung vorliegen muss.
%, wenn die aufgetrennten Stoffe anschließend weiter verwendet werden sollen. 
%Mit Hilfe des Verfahrens kann ein Gemisch aufgetrennt und nur der gewünschte, nun reine Teil davon genutzt werden. 

Eine andere Nutzung liegt in der Analyse von Stoffgemischen. Als Ergebnis einer solchen Analyse entsteht ein Chromatogramm, welches die aufgetrennten Stoffe visualisiert. Dieses enthält einige Peaks, von denen jeder für eine Stoffsorte steht. Wenn bekannt ist, wie sich unterschiedliche Stoffe bei der Chromatographie verhalten und wie sie sich im Chromatogramm zeigen, können daraus Rückschlüsse auf die im Ausgangsgemisch enthaltenen Stoffe gezogen werden. 

Praktische Anwendungen dafür finden sich in vielen Bereichen der Chemie und Biochemie. 
Um nur zwei Beispiele zu nennen, kann in der Analyse von Lebensmitteln mit Hilfe der Chromatographie bestimmt werden, ob Produkte Inhaltstoffe enthalten,  die nicht darin enthalten sein sollten. Auch in der Umweltchemie wird das Verfahren genutzt, um beispielsweise festzustellen, ob in einer Wasser- oder Luftprobe bestimmte, möglicherweise gefährliche, Stoffe enthalten sind.

% Auf diese Weise können 

% Beispielsweise können diese Stoffgemische dadurch analysiert werden. 
% 
% Kurzbeschreibung: Phasen, Chromatogramm aus Peaks.
% 
  
%können aus einem Stoffgemisch diejenigen Stoffe  dafür aber aus einem Stoffgemisch nur einige Stoffe genutzt werden sollen. 

Es gibt verschiedene Arten der Chromatographie, die im Laufe des letzten Jahrhunderts entwickelt wurden. Für diese Arbeit besondern interessant ist die Gaschromatographie in Kapillarsäulen. Die Gaschromatographie gibt es seit den 1950er Jahren, die Kapillartechnik wurde etwa zwanzig Jahre später erfunden.
Bei der Gaschromatographie ist die mobile Phase stets ein Gas welches das zu analysierende Stoffgemisch durch eine Trennsäule, beispielsweise eine Kapillarsäule, leitet. In diesem Fall ist die Säule innen mit einer flüssigen stationären Phase beschichtet. Eine Multikapillarsäule, kurz MCC (engl. Multi Capillary Column) besteht aus sehr vielen dieser Kapillaren.

% \paragraph{Multikapillarsäule} sollte oben schon...
% Eine Multikapillarsäule, kurz MCC (engl. Multi Capillary Column) ist eine Trennsäule in der Gaschromatographie. Eine kurze Einführung zur Chromatographie wird im nächsten Kapitel gegeben. An dieser Stelle sei nur gesagt, dass es sich dabei um ein Verfahren  handelt. Unter anderem kann die Chromatographie in einer MCC statt finden. Eine Kapillare ist dabei eine extrem dünne, innen beschichtete, Röhre, durch die mit Hilfe eines Trägergases das zu trennende Stoffgemisch geleitet wird. Die Analytteilchen interagieren mit der Beschichtung der Röhre und werden dadurch aufgetrennt. Eine Multikapillarsäule besteht aus sehr vielen dieser einzelnen Kapillaren.

%\todo{Das ist der Hintergrund, aber die Motivation?} 

%(Was ich nicht tue, was es schon gibt?, einfaches Modell)

\section{Ziel der Arbeit}

Im Rahmen der Diplomarbeit soll, mit Hilfes eines Modells, der chromatographische Prozess in einer Multikapillarsäule simuliert werden. 
Dabei soll jeweils nur ein Stoff, charakterisiert durch einige Parameter, simuliert werden. Das Resultat einer Simulation ist jeweils ein einzelner Peak. Sowohl diese simulierten Peaks, als auch die aus realen Messungen können durch bestimmte Eigenschaften charakterisiert werden. 

Gesucht ist nun eine Entsprechung dieser Peakcharakteristika
%, welcher durch verschiedene Maße beschrieben werden kann. S
%Ziel dieser Diplomarbeit ist es, mit Hilfe eines Modells den chromatographischen Prozess zu simulieren.
%Anschließend soll eine Entsprechung dieser Peakcharakteristika, wie sie in Kapitel \ref{chapter:gru} beschrieben werden,
zu den Simulationsparametern, die für das Modell verwendet werden. Allerdings ist zunächst nicht klar, ob eine solche Entsprechung zwischen Parametern und Charakteristika existiert. Wenn dies aber der Fall ist,  
%, gefunden werden, falls es eine solche Entsprechung gibt.
soll es sowohl möglich sein, aus den Simulationsparametern vorherzusagen, welche Eigenschaften ein damit simulierter Peak haben wird. Umgekehrt soll aber auch zu einem vorgegebenen Peak bestimmt werden können, welche Parameter nötig sind, um mit einer Simulation einen möglichst ähnlichen Peak zu erzeugen.

Es muss noch der Begriff der Simulation für diese Arbeit geklärt werden. Meist geht es bei einer Simulaiton um die Nachahmung realer Phänomene durch Hilfsmittel wie einen Computer. Im Zusammenhang mit der Gaschromatographie könnte man darunter beispielsweise verstehen, den physikalischen Ablauf eines Säulendurchlaufs auf Molekülebene durchzugehen. Dabei müssten Teilchen simuliert werden, die unter Anwendung physikalischer Gesetze miteinander interagieren, beispielweise aneinander anstoßen und dadurch Richtung und Geschwindigkeit ändern oder sich gegenseitig anziehen oder abstoßen und dabei ebenfalls ihre Bewegung ändern. Auch für Interaktionen mit der chromatographischen Apparatur müssten die Auswirkungen auf die Teilchen berechnet und eingebunden werden. 

Doch eine solche physikalische Simulation ist in diesem Fall genausowenig gemeint, wie die beiden Ansätze, die in Abschnitt \ref{chapter:ein_andere} vorgestellt werden.

Statt dessen soll hier mit Hilfe eines Modell der chromatographische Prozess simuliert werden. Es wird dabei von den Details, die bei einer Simulation auf physikalischer Ebene berücksichtigt werden müsste, abstrahiert und nur das grundlegende Prinzip des Wechsels der Teilchen zwischen zwei Phasen berücksichtigt. 

Das Problem kann daher als unbekannte Funktion $F: [0,1] ^ x \rightarrow \mathbb{R}^y$ formuliert werden. Dabei ist $x$ eine zunächst unbekannte Anzahl an Modellparametern $p$ für die Simulation und $y$ die Anzahl an Peakcharakteristika $\lambda$ mit denen ein Peak beschrieben wird. Schön wäre es, $F(p_1, \ldots, p_x) = (\lambda_1, \ldots, \lambda_y)$ für alle in echten Daten zu findenden $\lambda_j$ zu kennen. Zumindest aber soll geklärt werden welcher Parameter $p_i$ welchen Einfluss auf die verschiedenen $\lambda_j$ hat. 

%zu einem gegebenen Peak die Parameter bestimmen, mit denen eben jener Peak simuliert werden kann. Umgekehrt soll zu gegebenen Simulationsparametern, der dadurch erzeugte Peak vorhergesagt werden können.
%Umgekehrt soll es möglich sein, für eine gegebene Peaklage und Peakform die nötigen Simulationsparameter zu ermitteln, mit denen ein solcher Peak simuliert werden kann.

%\todo{Umformulierung auf hypothetische Peaks}
%Zum Vergleich liegen einige MCC-IMS-Messungen von Mischungen sieben bekannter Stoffe vor. Diese Datensätze wurden zur Verfügung gestellt von der Firma B \& S Analytik  (\mbox{\url{http://www.bs-analytik.de/}}). Zu den dadurch gegebenen Peaks sollen Parameter für die Simulation ermittelt und diese Peaks simuliert werden.
%Es geht dabei nicht um eine physikalische Simulation auf molekularer Ebene, sondern um die Entwicklung eines abstrakten, probabilistischen Modells, welches mit möglichst wenig Parametern auskommt. 


Um dieses Ziel zu erreichen, werden zunächst die Eigenschaften, die ein Peak haben kann, beschrieben. Dann werden mit einem sehr einfachen Modell Simulationen durchgeführt und evaluiert, inwiefern die damit erzeugten Peaks mit realen Peakdaten übereinstimmen und ob es Eigenschaften gibt, die nicht simuliert werden können. Ist dies der Fall, wird das Modell nach Bedarf angepasst oder erweitert. Auch das neue Modell wird getestet und evaluiert. 

%welches nach und nach angepasst werden muss, bis die durch die Referenzdatensätze vorgegebenen Peaks ausreichend angenähert werden können.
% 
% \todo{Vorgehensweise graphisch darstellen!} 
% \begin{enumerate}
%  \item Start mit einfachem 2-Zustände-Modell
%  \item Experimentelle Arbeit: Simulation mit verschiedenen Parametern
%  \item Überprüfung, ob Referenzpeaks angenähert werden können
%  \item Verfeinerung/Erweiterung des Modells
%  \item Wiederholung von 2-4 bis gegebene Peaks ausreichend angenähert
% \end{enumerate}


\section{Andere Arbeiten}
\label{chapter:ein_andere}

Es existieren bereits Simulatoren für die Chromatographie, welche jedoch Daten auf eine andere Art erzeugen als eine Simulation des chromatographischen Prozesses erzeugen.

Es gibt beispielsweise einen älteren Ansatz \citep{spreadsheet}, bei dem verschiedene Peaks als Funktionen eingegeben werden können. Die dort zu Grunde liegende Annahme ist, dass sich Peaks, ebenso wie das Hintergrundrauschen einer Messung, durch jeweils eine Funktion darstellen lassen. Diese Eingaben werden dann mit Hilfe einer Tabellenkalkulationssoftware zu einem gemeinsamen Spektrum kombiniert. Ein solcher Ansatz ist aber erst nützlich, wenn die Peaks bereits als Funktionen vorliegen, die erhofften Zusammenhänge zwischen Parametern einzelner Peaks und damit verbundenen Eigenschaften müssen also bereits im Vorhinein vorliegen.

Ein anderer Ansatz wurde von \citet{hplcsim} beschrieben. Dort geht es zwar um Flüssigchromatographie, die jedoch der Gaschromatographie ausreichend ähnlich ist, um als Vergleich herangezogen zu werden. In der dort beschriebenen Software können verschiedene Parameter der Chromatographie wie Druck oder Temperatur variiert und verschiedene Analyte ausgewählt werden und es werden daraus Chromatogramme erzeugt. Die so gewonnenen Daten werden aber nicht simuliert, sondern aus experimentellen Daten interpoliert. Dazu wurden in vielen Experimenten die Einflüsse der Einstellungen des Chromatographiegerätes untersucht und festgestellt, wie sich beispielsweise die Temperatur auf die resultierenden Peaks auswirkt. Für diese unterschiedlichen Messbedingungen wurden Formeln entwickelt, die in der Simulation auf die Peaks der einzelnen Analyte angewendet werden, um so je nach gewählten Einstellungen modifzierte Chromatogramme zu erhalten.


\section{Aufbau der Arbeit}
Zunächst werden in Kapitel \ref{chapter:gru} die nötigen Grundlagen für diese Arbeit vorgestellt. Dies ist zum Einen die Chromatographie, insbesondere die Gaschromatographie und der chromatographische Prozess. Zudem werden einige Eigenschaften, welche die resultierenden Peaks haben können, vorgestellt. Zum Anderen wird eine Möglichkeit zur Modellierung des Problems als Probabilistischer Arithmetischer Automat (PAA) in Betracht gezogen.

Kapitel \ref{chapter:mod} stellt zwei aufeinander aufbauende Modelle für die Simulation vor. Das erste ist ein sehr einfaches Modell, welches mit nur zwei Parametern auskommt, jedoch nur einen Teil der gewünschten Ergebnisse liefert. Das zweite Modell berücksichtigt weitere Details der Chromatographie und verfügt über bis zu sechs Parameter. Für beide Varianten wird jeweils auch eine Modellierung als PAA vorgestellt.

In Kapitel \ref{chapter:meth} werden Methoden zur Umsetzung der Modelle erklärt. Dabei gibt es zwei Herangehensweisen, die für jeweils das einfache und das komplexere Modell vorgestellt werden, zudem wird auch eine mögliche Umsetzung des PAA besschrieben.

Kapitel \ref{chapter:imp} gibt einen Überblick über die Implementierung der vorgestellten Methoden. 

In Kapitel \ref{chapter:eva} werden die Ergebnisse der Simulationen ausgewertet. Für beide Modelle wird der Einfluss der einzelnen Parameter auf die Simulationsergebnisse untersucht und mögliche zusammen wirkende Einflüsse der Parameter beleuchtet. Außerdem werden die Grenzen der Modelle aufgezeigt.

Zuletzt werden in Kapitel \ref{chapter:dis} die Ergebnisse der Arbeit zusammengefasst und ein Ausblick auf mögliche weiterführende Ideen oder Verbesserungen gegeben.