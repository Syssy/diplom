% einleitung.tex
\label{chapter:ein}

\chapter{Einleitung}

\todo{(Wo) Soll die Aufgabenstellung noch mal rein? Also Entsprechung zwischen Peakparametern und Sim-Parametern}
\section{Motivation und Hintergrund}
``Multikapillarsäule'', MCC (engl. Multi Capillary Column)

Trennsäule in der Gaschromatographie

%(Kurz anreissen, später mehr)
``Simulation''

Keine physikalische Simulation der Moleküle

 Keine Interpolation vorhandender Messungen

 Kein Überlagern verschiedener Kurven, um Gesamt-Spektrum zu erhalten
 
 sondern: Modell für chromatographischen Prozess

%(Was ich nicht tue, was es schon gibt?, einfaches Modell)


Im Rahmen der Diplomarbeit soll eine Multikapillarsäule simuliert werden. 

Es geht dabei nicht um eine physikalische Simulation auf molekularer Ebene, sondern um die Entwicklung eines abstrakten, probabilistischen Modells, welches mit möglichst wenig Parametern auskommt. 

In der Simulation soll jeweils nur ein Stoff, charakterisiert durch diese Parameter, simuliert werden. Das Resultat einer Simulation ist jeweils ein einzelner Peak, welcher durch seine Lage und Form beschrieben werden kann. Die Lage entspricht dabei der Retentionszeit am Maximum des Peaks. Die Form ist durch seine Breite an einer bestimmten Höhe (zum Beispiel Halbwertsbreite) sowie Schiefe gekennzeichnet.



\section{Aufbau der Arbeit}
In Kapitel \ref{chapter:gru} wird zu Einen die Chromatographie, insbesondere die Gaschromatographie vorgestellt und der chromatographische Prozess erklärt. Zum Anderen PAA

\section{Ziel der Arbeit}
\todo{Ziel und Vorgehensweise}
Umgekehrt soll es möglich sein, für eine gegebene Peaklage und Peakform die nötigen Simulationsparameter zu ermitteln, mit denen ein solcher Peak simuliert werden kann.

\todo{Umformulierung auf hypothetische Peaks}
%Zum Vergleich liegen einige MCC-IMS-Messungen von Mischungen sieben bekannter Stoffe vor. Diese Datensätze wurden zur Verfügung gestellt von der Firma B \& S Analytik  (\mbox{\url{http://www.bs-analytik.de/}}). Zu den dadurch gegebenen Peaks sollen Parameter für die Simulation ermittelt und diese Peaks simuliert werden.

Gesucht ist letztendlich eine allgemeine Entsprechung der Simulationsparameter zu den Parametern mit denen ein Peak beschrieben werden kann, falls diese existiert. Eine Berechnung des Peaks nur durch die Simulationsparameter und umgekehrt eine Vorhersage der Simulationsparameter zu bekannten Peakdaten soll damit realisiert werden.

Ziel dieser Diplomarbeit ist es, eine Entsprechung von Peakcharakteristika, wie sie in Kapitel \ref{chapter:gru} beschrieben werden, zu den Simulationsparametern, die für das endgültige Modell verwendet werden, zu finden, falls es eine solche Entsprechung gibt.
Das heißt es soll sowohl möglich sein, zu einem gegebenen Peak die Parameter bestimmen, mit denen eben jener Peak simuliert werden kann. Umgekehrt soll zu gegebenen Simulationsparametern, der dadurch erzeugte Peak vorhergesagt werden können.
 
Habe unbekannte Funktion : $F: [0,1] ^ x \rightarrow \mathbb{R}^y$. Dabei ist $x$ eine zunächst unbekannte Zahl nötiger Parameter und $y$ die Anzahl gefundener Peakcharakteristika. Schön wäre es, $F(p_1, \ldots, p_x) = (\lambda_1, \ldots, \lambda_y)$ für alle in echten Daten zu findenden $\lambda_j$ zu kennen. Zumindest aber soll geklärt werden welcher Parameter $p_i$ welchen Einfluss auf die verschiedenen $\lambda_j$ hat. 

Um dieses Ziel zu erreichen, wird zunächst mit einem sehr einfachen Modell begonnen, welches nach und nach angepasst werden muss, bis die durch die Referenzdatensätze vorgegebenen Peaks ausreichend angenähert werden können.

\todo{Vorgehensweise graphisch darstellen!} 
\begin{enumerate}
 \item Start mit einfachem 2-Zustände-Modell
 \item Experimentelle Arbeit: Simulation mit verschiedenen Parametern
 \item Überprüfung, ob Referenzpeaks angenähert werden können
 \item Verfeinerung/Erweiterung des Modells
 \item Wiederholung von 2-4 bis gegebene Peaks ausreichend angenähert
\end{enumerate}

\section{Andere Arbeiten}

Es existieren bereits Simulatoren für die Chromatographie, welche jedoch Daten auf eine andere Art erzeugen als eine Simulation des chromatographischen Prozesses erzeugen.

Es gibt beispielsweise einen älteren Ansatz \citep{spreadsheet}, bei dem verschiedene Peaks als Funktionen eingegeben werden können. Die dort zu Grunde liegende Annahme ist, dass sich Peaks, ebenso wie das Hintergrundrauschen einer Messung, durch jeweils eine Funktion darstellen lassen. Diese Eingaben werden dann mit Hilfe einer Tabellenkalkulationssoftware zu einem gemeinsamen Spektrum kombiniert. Ein solcher Ansatz ist aber erst nützlich, wenn die Peaks bereits als Funktionen vorliegen.

Ein anderer Ansatz wurde von \citet{hplcsim} beschrieben. Dort geht es zwar um Flüssigchromatographie, die jedoch der Gaschromatographie ausreichend ähnlich ist, um als Vergleich herangezogen zu werden. In der dort beschriebenen Software können verschiedene Parameter der Chromatographie wie Druck oder Temperatur variiert und verschiedene Analyte ausgewählt werden und es werden daraus Chromatogramme erzeugt. Die so gewonnenen Daten werden aber nicht simuliert, sondern aus experimentellen Daten interpoliert. Dazu wurden in vielen Experimenten die Einflüsse der Einstellungen des Chromatographiegerätes untersucht und festgestellt, wie sich beispielsweise die Temperatur auf die resultierenden Peaks auswirkt. Für diese unterschiedlichen Messbedingungen wurden Formeln entwickelt, die in der Simulation auf die Peaks der einzelnen Analyte angewendet werden, um so je nach gewählten Einstellungen modifzierte Chromatogramme zu erhalten.

 