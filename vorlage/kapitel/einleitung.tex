% einleitung.tex
\label{chapter:ein}

\chapter{Einleitung}

%\section{Motivation und Hintergrund}

Die Chromatographie ist ein Verfahren in der Chemie, zur Auftrennung von Stoffgemischen. Beispielsweise können diese Stoffgemische dadurch analysiert werden. 

Kurzbeschreibung: Phasen, Chromatogramm aus Peaks.

Wenn bekannt ist, wie sich unterschiedliche Stoffe bei der Chromatographie verhalten und welche Auswirkungen sie auf das Chromatogramm, das Ergebnis der Chromatographie haben, können daraus Rückschlüsse auf die in einem Stoffgemisch enthaltenen Stoffe gezogen werden. 

Praktische Anwendungen finden sich in vielen Bereichen der Chemie und Biochemie. 
Um nur zwei Beispiele zu nennen, kann in der Analyse von Lebensmitteln: ist Inhaltstoff enthalten, der nicht reingehört? Hinweise darauf im Chromatogramm
Anderer Bereich: Umweltchemie: Sindd bestimmte (gefährliche) Stoffe in Luft/Wasser enthalten?
  
Andere Anwendung der Auftrennung von Stoffen ist die Präparation, wenn die Stoffe nach der Chromatographie weiter verwendet werden sollen, dafür aber aus einem Stoffgemisch nur einige Stoffe genutzt werden sollen.

Geschichte: Erfindung/erste Erwähnung: durch den russischen Botaniker Tswett. Gaschromatographie in Kapillaren seit den 1970ern

\todo{Das ist der Hintergrund, aber die Motivation?}

%(Was ich nicht tue, was es schon gibt?, einfaches Modell)


Es geht dabei nicht um eine physikalische Simulation auf molekularer Ebene, sondern um die Entwicklung eines abstrakten, probabilistischen Modells, welches mit möglichst wenig Parametern auskommt. 

In der Simulation soll jeweils nur ein Stoff, charakterisiert durch diese Parameter, simuliert werden. Das Resultat einer Simulation ist jeweils ein einzelner Peak, welcher durch seine Lage und Form beschrieben werden kann. Die Lage entspricht dabei der Retentionszeit am Maximum des Peaks. Die Form ist durch seine Breite an einer bestimmten Höhe (zum Beispiel Halbwertsbreite) sowie Schiefe gekennzeichnet.
Im Rahmen der Diplomarbeit soll eine Multikapillarsäule simuliert werden. Dazu müssen zunächst die beiden Begriffe ``Multikapillarsäule'' und ``Simulation'' geklärt werden.
\paragraph{Multikapillarsäule} sollte oben schon...
Eine Multikapillarsäule, kurz MCC (engl. Multi Capillary Column) ist eine Trennsäule in der Gaschromatographie. Eine kurze Einführung zur Chromatographie wird im nächsten Kapitel gegeben. An dieser Stelle sei nur gesagt, dass es sich dabei um ein Verfahren  handelt. Unter anderem kann die Chromatographie in einer MCC statt finden. Eine Kapillare ist dabei eine extrem dünne, innen beschichtete, Röhre, durch die mit Hilfe eines Trägergases das zu trennende Stoffgemisch geleitet wird. Die Analytteilchen interagieren mit der Beschichtung der Röhre und werden dadurch aufgetrennt. Eine Multikapillarsäule besteht aus sehr vielen dieser einzelnen Kapillaren.

Als Resultat einer Chromatogramm entsteht ein blubb TODO %TODO


\section{Ziel der Arbeit}

%(Kurz anreissen, später mehr)
Ziel dieser Diplomarbeit ist es, mit Hilfe eines Modells den chromatographischen Prozess zu simulieren. Anschließend soll eine Entsprechung von Peakcharakteristika, wie sie in Kapitel \ref{chapter:gru} beschrieben werden, zu den Simulationsparametern, die für das Modell verwendet werden, gefunden werden, falls es eine solche Entsprechung gibt.

Mit dem Begriff Simulation können mehrere Dinge gemeint sein, meist geht es um die Nachahmung realer Phänomene durch Hilfsmittel wie einen Computer. Im Zusammenhang mit der Gaschromatographie könnte man darunter beispielsweise verstehen, den physikalischen Ablauf eines Säulendurchlaufs auf Molekülebene durchzugehen. Dabei müssten Teilchen simuliert werden, die unter Anwendung physikalischer Gesetze miteinander interagieren, beispielweise aneinander anstoßen und dadurch Richtung und Geschwindigkeit ändern oder sich gegenseitig anziehen oder abstoßen und dabei ebenfalls ihre Bewegung ändern. Auch für Interaktionen mit der chromatographischen Apparatur müssten die Auswirkungen auf die Teilchen berechnet und eingebunden werden. 

Doch eine solche physikalische Simulation ist in diesem Fall genausowenig gemeint, wie die beiden Ansätze, die in Abschnitt \ref{chapter:ein_andere} vorgestellt werden.

Statt dessen soll hier mit Hilfe eines Modell für chromatographischen Prozess simuliert werden. Es wird dabei von den Details, die bei einer Simulation auf physikalischer Ebene berücksichtigt werden müsste, abstrahiert und nur das grundlegende Prinzip des Wechsels der Teilchen zwischen zwei Phasen berücksichtigt werden.



Das heißt es soll sowohl möglich sein, aus den Simulationsparametern vorherzusagen, welche Eigenschaften ein damit simulierter Peak haben wird. Umgekehrt soll aber auch zu einem vorgegebenen Peak bestimmt werden können, welche Parameter nötig sind, um mit einer Simulation einen möglichst ähnlichen Peak zu erzeugen.

Gesucht ist also eine allgemeine Entsprechung der Simulationsparameter zu den Parametern mit denen ein Peak beschrieben werden kann, falls diese existiert. 
%Eine Berechnung des Peaks nur durch die Simulationsparameter und umgekehrt eine Vorhersage der Simulationsparameter zu bekannten Peakdaten soll damit realisiert werden.

Das Problem kann daher als unbekannte Funktion : $F: [0,1] ^ x \rightarrow \mathbb{R}^y$ formuliert werden. Dabei ist $x$ eine zunächst unbekannte Zahl nötiger Parameter in der Simulation und $y$ die Anzahl an Peakcharakteristika mit denen ein Peak beschrieben wird. Schön wäre es, $F(p_1, \ldots, p_x) = (\lambda_1, \ldots, \lambda_y)$ für alle in echten Daten zu findenden $\lambda_j$ zu kennen. Zumindest aber soll geklärt werden welcher Parameter $p_i$ welchen Einfluss auf die verschiedenen $\lambda_j$ hat. 

%zu einem gegebenen Peak die Parameter bestimmen, mit denen eben jener Peak simuliert werden kann. Umgekehrt soll zu gegebenen Simulationsparametern, der dadurch erzeugte Peak vorhergesagt werden können.
%Umgekehrt soll es möglich sein, für eine gegebene Peaklage und Peakform die nötigen Simulationsparameter zu ermitteln, mit denen ein solcher Peak simuliert werden kann.

%\todo{Umformulierung auf hypothetische Peaks}
%Zum Vergleich liegen einige MCC-IMS-Messungen von Mischungen sieben bekannter Stoffe vor. Diese Datensätze wurden zur Verfügung gestellt von der Firma B \& S Analytik  (\mbox{\url{http://www.bs-analytik.de/}}). Zu den dadurch gegebenen Peaks sollen Parameter für die Simulation ermittelt und diese Peaks simuliert werden.

Um dieses Ziel zu erreichen, wird zunächst mit einem sehr einfachen Modell begonnen. Es werden Simulationen mit diesem Modell durchgeführt und evaluiert. Anschließend wird das Modell nach Bedarf angepasst zu werden, falls die damit erzeugbaren Peaks, einige Eigenschaften realer Peaks nicht ausreichend annähern können. Auch das neue Modell wird getestet und evaluiert.
%welches nach und nach angepasst werden muss, bis die durch die Referenzdatensätze vorgegebenen Peaks ausreichend angenähert werden können.

\todo{Vorgehensweise graphisch darstellen!} 
\begin{enumerate}
 \item Start mit einfachem 2-Zustände-Modell
 \item Experimentelle Arbeit: Simulation mit verschiedenen Parametern
 \item Überprüfung, ob Referenzpeaks angenähert werden können
 \item Verfeinerung/Erweiterung des Modells
 \item Wiederholung von 2-4 bis gegebene Peaks ausreichend angenähert
\end{enumerate}


\section{Andere Arbeiten}
\label{chapter:ein_andere}

Es existieren bereits Simulatoren für die Chromatographie, welche jedoch Daten auf eine andere Art erzeugen als eine Simulation des chromatographischen Prozesses erzeugen.

Es gibt beispielsweise einen älteren Ansatz \citep{spreadsheet}, bei dem verschiedene Peaks als Funktionen eingegeben werden können. Die dort zu Grunde liegende Annahme ist, dass sich Peaks, ebenso wie das Hintergrundrauschen einer Messung, durch jeweils eine Funktion darstellen lassen. Diese Eingaben werden dann mit Hilfe einer Tabellenkalkulationssoftware zu einem gemeinsamen Spektrum kombiniert. Ein solcher Ansatz ist aber erst nützlich, wenn die Peaks bereits als Funktionen vorliegen, die erhofften Zusammenhänge zwischen Parametern einzelner Peaks und damit verbundenen Eigenschaften müssen also bereits im Vorhinein vorliegen.

Ein anderer Ansatz wurde von \citet{hplcsim} beschrieben. Dort geht es zwar um Flüssigchromatographie, die jedoch der Gaschromatographie ausreichend ähnlich ist, um als Vergleich herangezogen zu werden. In der dort beschriebenen Software können verschiedene Parameter der Chromatographie wie Druck oder Temperatur variiert und verschiedene Analyte ausgewählt werden und es werden daraus Chromatogramme erzeugt. Die so gewonnenen Daten werden aber nicht simuliert, sondern aus experimentellen Daten interpoliert. Dazu wurden in vielen Experimenten die Einflüsse der Einstellungen des Chromatographiegerätes untersucht und festgestellt, wie sich beispielsweise die Temperatur auf die resultierenden Peaks auswirkt. Für diese unterschiedlichen Messbedingungen wurden Formeln entwickelt, die in der Simulation auf die Peaks der einzelnen Analyte angewendet werden, um so je nach gewählten Einstellungen modifzierte Chromatogramme zu erhalten.


\section{Aufbau der Arbeit}
Zunächst werden in Kapitel \ref{chapter:gru} die nötigen Grundlagen für diese Arbeit vorgestellt. Dies ist zum Einen die Chromatographie, insbesondere die Gaschromatographie und der chromatographische Prozess. Zudem werden einige Eigenschaften, welche die resultierenden Peaks haben können, vorgestellt. Zum Anderen wird eine Möglichkeit zur Modellierung des Problems als Probabilistischer Arithmetischer Automat (PAA) in Betracht gezogen.

Kapitel \ref{chapter:mod} stellt zwei aufeinander aufbauende Modelle für die Simulation vor. Das erste ist ein sehr einfaches Modell, welches mit nur zwei Parametern auskommt, jedoch nur einen Teil der gewünschten Ergebnisse liefert. Das zweite Modell berücksichtigt weitere Details der Chromatographie und verfügt über bis zu sechs Parameter. Für beide Varianten wird jeweils auch eine Modellierung als PAA vorgestellt.

In Kapitel \ref{chapter:meth} werden Methoden zur Umsetzung der Modelle erklärt. Dabei gibt es zwei Herangehensweisen, die für jeweils das einfache und das komplexere Modell vorgestellt werden, zudem wird auch eine mögliche Umsetzung des PAA besschrieben.

Kapitel \ref{chapter:imp} gibt einen Überblick über die Implementierung der vorgestellten Methoden. 

In Kapitel \ref{chapter:eva} werden die Ergebnisse der Simulationen ausgewertet. Für beide Modelle wird der Einfluss der einzelnen Parameter auf die Simulationsergebnisse untersucht und mögliche zusammen wirkende Einflüsse der Parameter beleuchtet. Außerdem werden die Grenzen der Modelle aufgezeigt.

Zuletzt werden in Kapitel \ref{chapter:dis} die Ergebnisse der Arbeit zusammengefasst und ein Ausblick auf mögliche weiterführende Ideen oder Verbesserungen gegeben.