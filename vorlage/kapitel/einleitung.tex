% einleitung.tex
\label{chapter:ein}

\todo{(Wo) Soll die Aufgabenstellung noch mal rein? Also Entsprechung zwischen Peakparametern und Sim-Parametern}
\chapter{Einleitung}
\section{Motivation und Hintergrund}
``Multikapillarsäule'', MCC (engl. Multi Capillary Column)

Trennsäule in der Gaschromatographie

%(Kurz anreissen, später mehr)
``Simulation''

Keine physikalische Simulation der Moleküle

 Keine Interpolation vorhandender Messungen

 Kein Überlagern verschiedener Kurven, um Gesamt-Spektrum zu erhalten
 
 sondern: Modell für chromatographischen Prozess

%(Was ich nicht tue, was es schon gibt?, einfaches Modell)


\section{Aufbau der Arbeit}
Im Rahmen der Diplomarbeit soll eine Multikapillarsäule simuliert werden. 

Es geht dabei nicht um eine physikalische Simulation auf molekularer Ebene, sondern um die Entwicklung eines abstrakten, probabilistischen Modells, welches mit möglichst wenig Parametern auskommt. 

In der Simulation soll jeweils nur ein Stoff, charakterisiert durch diese Parameter, simuliert werden. Das Resultat einer Simulation ist jeweils ein einzelner Peak, welcher durch seine Lage und Form beschrieben werden kann. Die Lage entspricht dabei der Retentionszeit am Maximum des Peaks. Die Form ist durch seine Breite an einer bestimmten Höhe (zum Beispiel Halbwertsbreite) sowie Schiefe gekennzeichnet.

Umgekehrt soll es möglich sein, für eine gegebene Peaklage und Peakform die nötigen Simulationsparameter zu ermitteln, mit denen ein solcher Peak simuliert werden kann.

Zum Vergleich liegen einige MCC-IMS-Messungen von Mischungen sieben bekannter Stoffe vor. Diese Datensätze wurden zur Verfügung gestellt von der Firma B \& S Analytik  (\mbox{\url{http://www.bs-analytik.de/}}). Zu den dadurch gegebenen Peaks sollen Parameter für die Simulation ermittelt und diese Peaks simuliert werden.

Gesucht ist letztendlich eine allgemeine Entsprechung der Simulationsparameter zu den Parametern mit denen ein Peak beschrieben werden kann, falls diese existiert. Eine Berechnung des Peaks nur durch die Simulationsparameter und umgekehrt eine Vorhersage der Simulationsparameter zu bekannten Peakdaten soll damit realisiert werden.

In Kapitel \ref{chapter:gru} wird zu Einen die Chromatographie, insbesondere die Gaschromatographie vorgestellt und der chromatographische Prozess erklärt. Zum Anderen PAA


\section{Ziel der Arbeit}
\todo{Ziel und Vorgehensweise}

Entsprechung von Peakcharakteristika zu Simulationsparametern

Vorgehensweise:
\begin{enumerate}
 \item <+-> Start mit einfachem Modell
 \item <+->  Simulation mit verschiedenen Parametern
 \item <+-> Überprüfung, ob Referenzpeaks angenähert werden können
 \item <+-> Verfeinerung/Erweiterung des Modells
 \item <+-> Wiederholung von 2-4 bis ausreichend angenähert
\end{enumerate}