\chapter{Referenzdaten}
\label{chapter:ref}

\todo{alles zu den Referenzdaten}

\section{Daten}
Wo habe ich die Daten her, welche Messungen sind das, wie viele Peaks, welche vor/nachteile hat das ganze, 

Zum Vergleich der simulierten Peaks liegen einige MCC-IMS-Messungen von Mischungen sieben bekannter Stoffe vor. Diese Datensätze wurden zur Verfügung gestellt von der Firma B \& S Analytik  (\mbox{\url{http://www.bs-analytik.de/}}). 

\section{Gewinnung der Peakdaten}
Problem mit dem Rauschen, wo werden peaks abgeschnitten und warum genau da (wertehistogramm-methode). Je nach Lage des Peaks und mehr oder weniger gleichmäßigem Umgebungsrauschen wirken die Ergebnisse besser oder schlechter.
Dadurch: Je nach Ausschnitt unterschiedliche Daten
Handarbeit und Draufgucken
Problem, wenn mehrere Peaks zu einem Retentionszeitpunkt liegen, da diese im wertehistogramm überlagert werden, der kleinere Peak verschwindet auf jeden Fall, der größere ist wahrscheinlich verfälscht. Daher noch mehr Handarbeit, geeignete Umgebungen für jeden Peak zu finden
Nur wenige Messpunkte (2pro sekunde) daher viel ``ungenauer'' als meine sims