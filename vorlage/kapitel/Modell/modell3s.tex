\section{3-Zustände Modell} 
\label{chapter:mod:3s}

%TODO Der letzte Absatz zu weiteren möglichen Erweiterungen kann dann als kleiner absatz ganz ans ende des modellkapitels und wird im ausblick noch mal aufgegriffen

Bisher fand keine Unterscheidung zwischen Adsorption und Lösung der Teilchen an bzw. in der stationären Phase statt, sodass zwei Zustände und Transitionen zwischen diesen beiden Zuständen als Modell ausreichten. In der Realität können beide Wechselwirkungen parallel zueinander statt finden, wenn ein Teilchen in den stationären Zustand wechselt, wird jeweils entschieden, ob es adsorbiert wird oder sich löst.
Außerdem ist es möglich, dass sich die Wahrscheinlichkeiten, in einen der beiden Zustände überzugehen oder in die mobile Phase zurückzukehren, unterscheiden. Daher liegt es nahe, einen neuen Zustand einzuführen, sodass die Adsorption und die Lösung voneinander getrennt behandelt werden. Die Tatsache, dass Tailing, wie bereits anfangs erwähnt, unter anderem durch zusätzlich Adsorptionseffekte verursacht sein kann, lässt vermuten, dass durch diesen dritten Zustand ein Tailing in der Simulation verursacht werden kann. 
 
\begin{figure}[h]
\begin{subfigure}[t]{0.9\textwidth}
\centering
\begin{tikzpicture}[->,>=stealth',shorten >=1pt,auto,node distance=3cm,
  thick,main node/.style={circle,fill=white!20,draw,font=\sffamily\large\bfseries, minimum width = 25pt},
   knoten/.style={thick,fill=none,inner sep=0pt, minimum width=25pt, circle}]
{
  \node[main node] (m) at (0, 0) {m};
  \node[main node] (s1) at (5, 1.5) {a}; 
  \node[main node] (s2) at (5, -1.5) {l};

  \path[every node/.style={font=\sffamily}]
        
    (m)  edge [bend left] node[above left]{$p_{ma}$} (s1) 
            edge [bend right] node[below left]{$p_{ml}$} (s2)
             edge [loop left] node {$p_{mm}$} (m)

    (s2)   edge node [right] {$p_{lm}$} (m)
            edge [loop right] node {$p_{ll}$} (s1)
        
    (s1)  edge node [right] {$p_{am}$} (m)
	  edge [loop right] node {$p_{aa}$} (s1)

        ;}
\end{tikzpicture}
\hfill
\caption{Getrennte Zustände}
\label{tikz:4p_Mod_a}
\end{subfigure}

\begin{subfigure}[t]{0.9\textwidth}
\centering
\begin{tikzpicture}[->,>=stealth',shorten >=1pt,auto,node distance=3cm,
  thick,main node/.style={circle,fill=white!20,draw,font=\sffamily\large\bfseries, minimum width = 25pt}]
{
   \node[main node] (m) at (0,0) {m}; 
  \node[main node] (s1) at (4,0) {a};
  \node[main node] (s2) at (8,0) {l};

  \path[every node/.style={font=\sffamily}]

    (s2)   edge [bend left] node [below] {$p_{la}$} (s1)
            edge [loop right] node {$p_{ll}$} (s1)
        
    (s1)   edge [bend left] node [above] {$p_{al}$} (s2)
            edge [bend left] node[below] {$p_{am}$} (m)
			edge [loop above] node {$p_{aa}$} (s1)
        
    (m)  edge [bend left] node[above]{$p_{ma}$} (s1)
             edge [loop left] node {$p_{mm}$} (m)
        ;}
\end{tikzpicture}

\caption{Zwischenzustand}
\label{tikz:4p_Mod_b}
\end{subfigure}
\caption{Mögliche weitere Modelle}
\label{tikz:4p_Mod}
\end{figure}
 
In Abbildung \ref{tikz:4p_Mod} sind zwei Möglichkeiten gezeigt, dem bisherigen Modell einen neuen Zustand hinzuzufügen.
Neben dem mobilen Zustand (m) gibt es in beiden Fällen je zwei stationäre Zustände, einen für die Adsorption (a) und einen für die Lösung (l). Die Übergangswahrscheinlichkeiten zwischen dem alten Zustand $i$ und dem neuen Zustand $j$ sind jeweils $p_{ij}$. 

In Abbildung \ref{tikz:4p_Mod_a} sind die beiden stationären Zustände getrennt voneinander, dazwischen finden keine direkten Übergänge statt. 
Anschaulich kann man sich vorstellen, dass bei Übertritt in die stationäre Phase schon festgelegt wird, welcher Art der Übergang sein wird. Es scheint intuitiv sinnvoll zu sein, die Übergangswahrscheinlichkeiten zu den beiden Zuständen sowie die beiden Wahrscheinlichkeiten, wieder in die mobile Phase einzutreten, sehr unterschiedlich zu wählen. Damit wird bezweckt, dass sie sich nicht einfach wieder zu einer Gesamtwahrscheinlichkeit aufaddieren, die man auch mit dem 2-Parameter Modell hätte erreichen können. In einen der Zustände sollen die Teilchen also nur seltener kommen, dafür dort sehr lange verweilen,  in den anderen Zustand wechseln die Teilchen häufiger, bleiben aber auch nicht so lange. Außerdem spiegelt dies die in Kapitel \ref{chapter:gru} erwähnte Tatsache wieder, dass adsorpierte Teilchen sich leichter wieder in die mobie Phase begeben können, als dies bei gelösten Teilchen der Fall ist.

In Abbildung \ref{tikz:4p_Mod_b} dient der erste stationäre Zustand als Übergangszustand zum zweiten stationären Zustand. Dabei ist der Übergangszustand die Adsorption als Hinweis darauf, dass die Teilchen, die sich in der stationären Phase lösen, zunächst mit deren Oberfläche in Kontakt treten und zumindest kurzzeitig adsorbiert sind, gleiches gilt für die andere Richtung. Nachdem ein Teilchen adsorbiert wurde, besteht die Möglichkeit, dass es sich noch in der stationären Phase löst oder auch adsorbiert bleibt. Auch hier liegt es nahe, die Wahrscheinlichkeiten für die verschiedenen Übergänge so zu gestalten, dass die Teilchen in einem der stationären Zustände seltener aber länger bleiben.

Beide Fälle können zu einem Gesamtmodell mit drei Zuständen zusammengefasst werden. Das Modell hat die Zustände $\mathcal{Q} = \{m, a, l\}$. $m$ ist wie auch im 2-Parameter Modell der Startzustand. Aus den oben angegebenen Übergangswahrscheinlichkeiten ergibt sich dann die Transitionsmatrix 
$T= 
\begin{pmatrix}
p_{mm} &  p_{ma} & p_{ml} \\
p_{am} &  p_{aa} & p_{al} \\
p_{lm} &  p_{la} & p_{ll} 
\end{pmatrix}
$  

In Modell 3a sind die Wahrscheinlichkeiten für $p_{al}$ und $p_{la}$ immer $0$ und in 3b gilt dies entsprechend für $p_{ml}$ und $p_{lm}$.

\subsection{PAA}

Analog zum PAA für nur zwei Zustände kann für drei Zustände ein PAA definiert werden. 
\todo{3sPAA aufschreiben}

\todo{Bild 3sPAA?}

\todo {kommt diese berechnung nicht eher in die methoden?}
Berechnung für nächsten Schritt '
Habe Matrix mit m, a, l und Orten 0...Länge mit den jeweiligen W'keiten
Neue Matrix:
$s^{'}_x(n) = s_{x-1}(1) * p_{1,n} + \sum^{|N|}_{m=2}{s_x(m) * p_{m,n}}$


\section{Weitere Modelle}

Eine weitere Möglichkeit, das Modell zu verändern, besteht darin, das Gleichgewicht, welches sich zwischen den beiden Phasen aufbaut, zu berücksichtigen. Damit müssten die Wahrscheinlichkeiten, den Zustand zu wechseln, nicht mehr fest vorgegeben sein, sondern sich dynamisch während der Simulation aus der aktuellen Verteilung der Teilchen an einem bestimmten Ort auf die Zustände berechnen.

Außerdem existieren Idee, noch eine Sättigung der freien Plätze zur Adsorption einzubauen

\todo{In Intro auf Gleichgewicht bei Adsoption eingehen, -> Modellerweiterung} 

% Erweiterung um weiteren (stat) Zustand um Unterschied zwischen Lösung und Adsorption zu haben
% Ideen zur Geschwindigkeit und warum das Quatsch ist. ?