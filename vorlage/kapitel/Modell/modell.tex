\chapter{Modell} %Modell/Modelle?
\label{chapter:mod}

\section{2-Parameter Modell}

Als Grundlage für die Entwicklung eines Simulationsmodells dient die Beobachtung des chromatographischen Prozesses. Dieser ist wie in TODO beschrieben gekennzeichnet durch viele Teilchen, welche häufig zwischen der stationären und mobilen Phase wechslen. Es liegt also nahe, dass im Modell ebenfalls Teilchen simuliert werden, die zwischen zwei Zuständen, welche die beiden Phasen repräsentieren, wechseln. Der Phasenwechsel geschieht jeweils mit einer bestimmten Wahrscheinlichkeit. Dabei kann es möglich sein, dass die Wahrscheinlichkeiten für den Wechsel in die eine oder andere Richtung voneinander abhängig sind, jedoch muss dies nicht sein, sodass für das Modell der allgemeinere Fall von unabhängigen Wahrscheinlichkeiten angenommen wird. Es sei also $p_s$ die Wahrscheinlichkeit, dass ein Teilchen, welches sich bereits in der stationären Phase befindet, auch stationär bleibt und $1-p_s$ die Wahrscheinlichkeit, dass es in die mobile Phase übergeht. Analog seien $p_m$ und $1-p_m$ die Wahrscheinlichkeiten, dass ein mobiles Teilchen in der mobilen Phase bleibt bzw. zur stationären Phase wechselt. 


Damit ergibt sich als Modell ein einfacher Automat mit zwei Zuständen $\mathcal{Q} = {s, m}$ \todo{Formale Beschreibung meines Modells}

Für die Simulation müssen x Teilchen den Automaten durchlaufen, dabei Ort festhalten für abbruchbedingung und zeit für peak
Das ganze 1000 mal wiederholen da je nur eine kapillare simuliert wurde

% Grundlage: Phasenwechsel
% Das eigentliche Modell
% Simulation je einer Kapillare und eines Stoffes -> Viele Durchläufe

\subsection{PAA zu diesem Modell}
% Modell als PAA
Blabla...

Der PAA ist formal definiert durch: $\mathcal{Q} = {s, m}$, $q_0 = m$, $T = 

\begin{figure}
 \centering
  \begin{tikzpicture}[->,>=stealth',shorten >=1pt,auto,node distance=3cm, thick,
   state node/.style={circle,fill=white,draw,font=\sffamily\bfseries, minimum height=30pt, text width = 1.6cm, align = center},
   operation node/.style= {regular polygon, regular polygon sides=3,  fill = white, draw, inner sep=  0pt},
   emission node/.style={rectangle, fill = lightgray!30, draw, text width = 1.8cm}]

       % \draw[fill=green] (current page.north west) rectangle (current page.south east);
%  
   \node[state node] (s) at (2, 1)  {stationär};
   \node[state node] (m) [right = of s] {mobil};
   \node[operation node, align = center] (os)  at (1.5,0.2) {$+$};
   \node[operation node, align = center] (om)  at (7.5,0.2) {$+$};

   \path[every node/.style={font=\sffamily\large}]

    (s)   edge [bend right] node [below] {$1-p_s$} (m)
            edge [loop left] node {$p_s$} (s)
        
    (m)  edge [bend right] node[above]{$1-p_m$} (s)
             edge [loop right] node {$p_m$} (m)
        ;

   \fill[color=lightgray!40] (1.1, 3.5) --(2,1.5) -- (2.9,3.5) -- cycle ;
   \fill[color=lightgray!40] (6.05, 3.5) -- (6.95,1.5) -- (7.85,3.5) -- cycle ;

   \node[emission node, align = center] (em) [above of = m] {$e_m(0) = 0$\\$e_m(1)=1$};
   \node[emission node, align = center] (es) [above of = s] {$e_s(0) = 1$\\$e_s(1)=0$};
  
  \end{tikzpicture}
  \caption{PAA für das 2-Parameter Modell} \label{PAA_2P}
\end{figure}

Und noch mehr Text, der sich auf \ref{PAA_2P} bezieht

\subsection{Probleme mit dem Modell}
% Probleme damit

\section{Weiteres Modell}
\todo{Erweiterung des Modells, wenn diese existiert}
% Erweiterung um weiteren (stat) Zustand um Unterschied zwischen Lösung und Adsorption zu haben
% Ideen zur Geschwindigkeit und warum das Quatsch ist. ?